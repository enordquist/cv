\documentclass{clean_cv}

% Add a BibTeX-style file encoding all of your publications to include here. You can export this from Zotero. Only include
% publications you want to appear here!
\addbibresource{publications.bib}


\begin{document}
\author{Erik Nordquist}
%\headlineposition{PhD Candidate}
\maketitle
% In this section, you can use any of the FontAwesome icons. The commands \faCenter and \faCenterStyle have been defined to properly center the icons
% when using the default font settings.
%
% You can use any of the icons listed in the fontawesome5 package documentation (https://ctan.math.utah.edu/ctan/tex-archive/fonts/fontawesome5/doc/fontawesome5.pdf)
% If you need to specify a specific style (as is done here for the address card), you should use the two-argument \faCenterCycle command
\begin{center}
\begin{tabular}{lll}
    %\faCenter{envelope} \href{mailto:erikbnordquist@gmail.com}{erikbnordquist@gmail.com}  & \faCenter{phone-alt} 217-381-4252 & \faCenterStyle{regular}{address-card} 1040 N Pleasant St Apt 406 Amherst MA 01002\\
    %\faCenter{orcid} \href{https://orcid.org/0000-0001-9139-0152}{0000-0001-9139-0152} & \faCenter{github} \href{https://github.com/enordquist}{enordquist} & \faCenter{globe} \url{https://eriknordquist.com} \\

    \faCenter{envelope} \href{mailto:erikbnordquist@gmail.com}{erikbnordquist@gmail.com}  & \faCenter{linkedin} \href{https://www.linkedin.com/in/erik-nordquist}{erik-nordquist} & \compchem {} \href{https://www.eriknordquist.com}{eriknordquist.com}
\end{tabular}
\end{center}

\vspace{-1.5em}

\section{Education}

% The datetabular environment takes one argument, which is the width of the left date column. As seen here:
%   9em is a good choice for "dual-date" formats (e.g. Sep 2015 - Nov 2019).
%   4em is a good choice for month/year dates (Sep 2014).
%   2em is a good choice for year-only dates (as seen in the publications)
\begin{datetabular}{6em}
% This is just a tabular environment, for the most part. The dateentry command has been defined for
% convienence. It takes two arguments, the first is the date and the second is whatever you wish placed to the right.
\dateentry{--}{
\textbf{Ph.D. Chemistry}{, University of Massachusetts Amherst}
}
\dateentry{2018}{
\textbf{B.S. Chemistry and Physics}{, The College of Idaho}
}
\end{datetabular}

\section{Fellowships}
\begin{datetabular}{6em}
\dateentry{2020 -- 2022}{
\textbf{Chemistry-Biology Interface (CBI) Fellowship}

\href{https://cbi.chem.umass.edu}{Funded Traineeship through NIH and UMass}
}
\dateentry{2022}{
\textbf{UMass CNS Teaching Fellowship}

\href{https://blogs.umass.edu/applyteachfellows}{Funded Traineeship through UMass in connection with CIRTL}
}
% There is something seriously funky happening between the tabular commands and the end of the itemize blocks.
% The command \eatvspace should be used if extra space appears at the end of a datetabular environment.
\end{datetabular}

%%%%%%%%%%%%%%%%%%%%%%%%%%%%
%% Publications
%%%%%%%%%%%%%%%%%%%
\section{Publications}
\nocite{*} % Loads every entry from the attached .bib file
% Highlight takes three entries, given name, given name initials, and family name.
% If you have a middle initial, this call looks like:
% \highlightauthorname{Bob H.}{B. H.}{Smith}
\highlightauthorname{Erik}{E.}{Nordquist} 
\begin{datetabular}{6em}
%printbibyear has been defined to only print entries from a given citation year.
\dateentry{2021}{\printbibyear{2021}}
\dateentry{2020}{\printbibyear{2020}}
\end{datetabular}

%%%%%%%%%%%%%%%%%%%%%%%%%%%
%%% Teaching and Mentoring
%%%%%%%%%%%%%%%%%%
\section{Undergraduate Mentoring}
\begin{datetabular}{6em}
\dateentry{2019 -- 2020}{ Callie Jillson studied how DnaK's helical lid impacts substrate binding. }
\dateentry{2020 -- 2021}{ Samantha Schultz used protein-like nanopores to help design a protocol for accelerating the sampling of hydrophobic dewetting transitions. }
\end{datetabular}

%%%%%%%%%%%%%%%%%%%%%%%%%%%%
%%%% Relevant Skills
%%%%%%%%%%%%%%%%%%%%%%
%\section{Relevant Skills}
%\begin{datetabular}{6em}
%\dateentry{Software and methods}{Molecular dynamics (Charmm, Gromacs, Plumed), Umbrella sampling, Metadynamics, %Rosetta, Bash, Unix, Python, Pandas, Scikit-learn, Sage, Mathematica}
%\end{datetabular}

%%%%%%%%%%%%%%%%%%%%%%%%%%%
%%% Presentations
%%%%%%%%%%%%%%%%%%%%
\section{Presentations}
\begin{datetabular}{6em}

\dateentry{2022}{
\textbf{Biophysical Society Annual Meeting}
\newline{"Free Energy of Hydrophobic Dewetting in Gating of BK Channels"}
}

\dateentry{2020}{
\textbf{Northeastern Structural Symposium Research Talk}
\newline{"Physical Origins of Selective Promiscuity to Hsp70s Revealed Through Physics-Based Modeling"}
\newline{}%\newline{}
\textbf{UMass ResearchFest Poster}
\newline{W.E. McEwen Poster Prize winner; "Physical Origins of Selective Promiscuity to Hsp70s Revealed Through Physics-Based Modeling"}
}

\dateentry{2019}{
\textbf{Molecular Biophysics in the Northeast Poster Session}
\newline{"Understanding the Origins of DnaK's Selective Promiscuity with Physics-based Modeling"}
}
\end{datetabular}

%%%%%%%%%%%%%
%% Professional Development
%%%%%%%%%%%%%
\section{Professional Development}
\begin{datetabular}{6em}

\dateentry{2021}{
\textbf{\href{https://eriknordquist.com/posts/2021/10/IEBT}{Intro. to Evidence-based Undergraduate STEM Teaching}}
\newline{8-week CITRL-run course, incl. topics: active learning, formative assessments, backward design, etc.}
\newline \newline
\textbf{\href{https://www.inclusivestemteaching.org/}{The Inclusive STEM Teaching Project online course}}
\newline{4-week CIRTL-run, evidence-based course on inclusive STEM teaching.}
}

\dateentry{2019}{
\textbf{OpenACC GPU Hackathon at MIT}
\newline{Parallelizing implementation of implicit solvent model GBMV2/SA}
}
\end{datetabular}
%%%%%%%%%%%%%%%%%%%%%%%
%% Service
%%%%%%%%%
\section{Service}
\begin{datetabular}{6em}

\dateentry{2021}{
\textbf{Journal Referee}
\textbf{Biophysical Journal}{, reviewed 1 article}
}

\dateentry{2021}{
\textbf{Search Committee for Grad Program Director}{, UMass Amherst Chem. Dept.}
}
\dateentry{2021 --}{
\textbf{\href{https://www.emerginginvestigators.org/}{Journal for Emerging Investigators}}{, 11 articles reviewed to-date}
}

\dateentry{2020}{
\textbf{CBI Alumni Networking Event}{, Organization Committee}
}

\dateentry{2019 -- 2021}{
\textbf{Annual Chem Dept ResearchFest}{, Organization Committee}
}
\end{datetabular}

\end{document}
